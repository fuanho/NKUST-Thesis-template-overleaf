
\chapter{緒論}\label{introduction}


\section{前言}\label{preface}

NKUST LaTeX 論文版型提供給本校研究生撰寫論文。當您使用這個專案,表示您已經進入碩士生涯的最後階段,祝福您也恭喜您即將完成碩士學業。希望這個專案能在碩士生涯的尾聲助你一臂之力。

\textbf{\color{red} 如果你是使用 Overleaf 撰寫論文,可直接跳到 \ref{ch_tmp_config} 版型設定。}本專案所使用的工具皆為 open source 軟體,可放心地由網路上自由下載合法的免費使用,文件編譯工具皆採用 TUG(TEX Users Group) 提供的 TexLive 套件包,編輯器使用 Microsoft 的 VSCode,並以 GNU Bash 環境作為開發的基礎。



\newpage

\section{研究動機}\label{motive}

好玩。


\section{論文架構}\label{thesis_arch}
\n 本論文編排方式如下:

第\ref{introduction}章 緒論

第\ref{ch_enviornment}章 編輯環境

第\ref{ch_tmp_config}章 版型設定

第\ref{ch_how2start}章 使用指南

第\ref{ch_content}章 $LaTeX$ 文字範例

第\ref{algorithm}章 演算法虛擬碼範例

第\ref{Experimental_picture}章 圖表與圖片

第\ref{conclusion_and_future}章 結論
